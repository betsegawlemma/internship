\chapter{Introduction}
\label{chapter:intro}

\section{Context of the Study}	 
Provide a brief history of the issues to date.

Situate your particular topic within the broad area of research.

Note that the field is changing, and more research is required on your topic.
\section{Problem statement or Motivation for the Study}
Identify a key point of concern (for example, increasing use or prominence, lack of research to date, response to an agenda, a new discovery, or perhaps one not yet applied to this context).

Refer to the literature only to the extent needed to demonstrate why your project is worth doing. Reserve your full review of existing theory or practice for later chapters.


Be sure that the motivation, or problem, suggests a need for further investigation.

\section{Aim and Scope}
Be sure that your aim responds logically to the problem statement.

Stick rigorously to a single aim. Do not include elements in it that describe how you intend to achieve this aim; reserve these for a later chapter.

When you have written the conclusions to your whole study; check that they respond to this aim. If they don't, change the aim or rethink your conclusions.

If you change the aim, revise the motivation for studying it.

Be sure to establish the scope of your study by identifying limitations of factors such as time, location, resources, or the established boundaries of particular fields or theories.

\section{Significance of the Study}
Explain how your thesis contributes to the field.

There are four main areas of contribution: theory development, tangible solution, innovative methods, and policy extension. One of these contributions must be identified as the basis of your primary contribution to the field.

In contrast to reports for industry, theory development is an expected and required contribution; for PhDs in particular, it must be ``original''.

\section{Structure of the Thesis}
\label{section:structure} 

You should use transition in your text, meaning that you should help the reader follow the thesis outline. Here, you tell what will be in each chapter of your thesis. 

