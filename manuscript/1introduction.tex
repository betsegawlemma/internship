\chapter{Introduction}
\label{chapter:intro}
According to the report released from Cisco, "The Zettabyte Era", the number of networked device is expected to increase from 17.1 billion in 2016 to 27.1 billion in 2021. In another report titled "Cisco Global Cloud Index, 2015 to 2020", global cloud IP traffic will grow more than three times and, among all the workloads performed in data-centers, by the year 2020, 92\% of them will be performed in cloud data-centers. The remaining 8\% will be performed in traditional data-centers. In response to this growing trends, the data-centers are continuously expanding. This expansion raises a primary concern on the amount of energy required to support the added data-center components and the growing service demands. 

The energy consumption issue is further aggravated due to the fact that current servers and network devices are energy inefficient. Of the total power consumed by a given computing or communication device, the idle power consumption takes the greater proportion. An IT equipment is considered energy efficient when it consumes power proportional to the amount of computing or data transfer task it performs. Currently there are different techniques implemented at a device level to tackle the energy inefficiency problem. For computing device, for instance, the operating frequency of the CPU can be lowered when the amount of task reach below some threshold value. Similarly, for a communicating devices, the data transferring rate can be lowered depending on the traffic or the device can be set to sleep (low power mode) when there is no traffic. However, these techniques are not fully utilized, as the techniques induce performance penalty to switch from one mode to another.  

Solving the energy consumption issue is primarily driven by economical factor (to save energy consumption bills) and environmental factor (to reduce \($$CO_2$$\) emission). There are other factors also, such as reducing the heat generated by a given IT device. The more energy inefficient a device is the more heat it generates. This will affect the life time and proper functioning of the device. 

Currently researchers are tackling the energy inefficiency issue at different levels. At a hardware level, for instance, to find a new  energy saving technique or to optimize the existing ones. At a software level, it can range from finding energy aware routing algorithm for network devices to energy efficient load balancing and workload assignment of servers at infrastructure level. 

There are three approaches that are in common use for doing energy related research at the infrastructure level. The first approach is to experiments on a real network. Though in this approach one might get the most real picture of the situation at a given moment, it wold be very difficult to repeat the experiment due to the transient nature of the experimental parameters such as workload and network traffic on a real network.  Furthermore, the real or production network might not be available for experimentation. The second approach, is to use an experimental test-bed. This gives full control over the experiment parameters and is also available. However, when the platform being experimented becomes very large it would be infeasible due to the hardware purchasing cost. In addition, experimenting on a new hypothesis might require setting up a new test-bed with a new hardware and configuration. This is costly in terms of money and also in terms of time. The third approach is to use simulation software for experimentation. This approach gives the ultimate control, flexibility and scalability compared to the other two. The main challenge is accurately modeling the real characteristics of the components involved in the problem under investigation.

In the context of computer networking, we can classify simulators into two: packet-level and flow-level simulators. Packet-level simulators strive to capture fine-grain details of a given network. Flow-level simulators, on the other hand, use analytical equations that approximate the behavior of the phenomenon being modeled using few parameters. Compared to flow-level simulators, packet-level simulators are considered to be more accurate, due to the detailed information they capture. However, they fail to scale well due to the time and storage required to process and store the captured information. A typical example for the former  is NS-3, a packet-level network simulator and for the later is SimGrid, a large-scale distributed network simulator. 

Despite the advantages simulators can offer for studying the energy inefficiency problem that exist in large-scale networks, search of the literature revealed only few packet-level simulators proposed to address the issue. As we have mentioned earlier, packet-level simulators can not scale well in the area of large-scale networks. To our knowledge there is no flow-level simulator proposed that can simulate the energy consumption of computing and communication components of large-scale networks. 

Therefore, the purpose of this study is to investigate the level of accuracy and scalability of flow-level energy consumption models in estimating energy consumption of large-scale networks. To fulfill this purpose we use SimGrid simulator. SimGrid already have energy consumption model for computing components, our work is limited to adding flow-level simulation model for communicating devices such as switches. Further more, we are only concerned with wired network components and concepts. 

Our contribution in this work is two-fold. First the proposed flow-level simulator can be used by researchers to find solutions to the energy inefficiency problem. Second, to other researchers who are interested to do similar work, for instance, for the wireless network they can follow the method that we outlined during our work.  

The rest of this thesis is organized as follow. In Chapter~\ref{chapter:background}, we first describe relevant concepts in relation to our work and then review other related works. In Chapter~\ref{chapter:environment}, we explain about our experimental environment. Then in Chapter~\ref{chapter:methods}, we discuss by comparing the advantages and disadvantages of commonly used methods to approach similar studies and then we outline the method we followed. In Chapter~\ref{chapter:implementation} we present the implementation and then in Chapter~\ref{chapter:evaluation} we discuss about the (in)validation experiment we conducted to evaluate the implementation. Following the validation result, in Chapter~\ref{chapter:discussion} we analyze and discuss the result. Finally, in Chapter~\ref{chapter:conclusions} we give conclusion remarks and we also forward continuing works that interested researchers might want to know.


