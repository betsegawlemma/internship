\chapter{Introduction}
\label{chapter:intro}

\section{Context of the Study}	 
Provide a brief history of the issues to date.

Situate your particular topic within the broad area of research.

Note that the field is changing, and more research is required on your topic.
\section{Problem statement}
Even though there is a growing concern about the energy consumption of large-scale network infrastructures, a search of the literature revealed few studies which address the issue of estimating the energy consumption of this network infrastructures. The existing proposed solutions being packet-level estimators are not scalable (in-terms of memory usage and speed) to be used in the domain of large-scale distributed networks such as cloud and grid computing. 

\section{Aim and Scope}

The purpose of this work is to propose efficient and reasonably accurate flow-level model in the context of SimGrid simulator for estimating the energy consumption of large-scale distributed networks. 

We can put energy consuming components in a large-scale network infrastructure into two broad groups: IT equipments (which includes computing servers, storage servers and networking components) and infrastructure components (which includes power provisioning, cooling and lighting components). Since SimGrid already have energy consumption model for computing servers, the scope of this study is on the network component part of the IT equipments category.

\section{Significance of the Study}

Explain how your thesis contributes to the field.

There are four main areas of contribution: theory development, tangible solution, innovative methods, and policy extension. One of these contributions must be identified as the basis of your primary contribution to the field.

In contrast to reports for industry, theory development is an expected and required contribution; for PhDs in particular, it must be ``original''.

\section{Structure of the Thesis}
\label{section:structure} 

You should use transition in your text, meaning that you should help the reader follow the thesis outline. Here, you tell what will be in each chapter of your thesis. 

