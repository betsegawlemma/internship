\chapter{Evaluation}
\label{chapter:evaluation}
In this chapter we present the accuracy and the scalability experiments we performed to evaluate the flow-level implementation. As we outlined in our method chapter, we compare our implementation with ECOFEN.  
\section{Experiment Setup}

\subsection{General ECOFEN related setup}
Among the three power consumption models available in ECOFEN module we used the \emph{linear} one. This model accepts idle power consumption value in watts and also energy consumption value for each byte received or send in nano-joule unit value. Throughout our experiments we used 10.3581 watt as idle power consumption value and 3.423 nano-joule as byte energy consumption value. 

The linear model estimate power consumption value based on these two inputs and the amount of transmitted or received traffic and then display the estimated power consumption in watt at specified time interval. We set the interval to be 0.05 sec, which means we get 20 estimations during each simulated second. 

What we want to get from ECOFEN is the total amount of energy consumed by the simulated network. To get this value, we first compute the average power drawn by the network for a given data transfer task and when the task ends, we multiply the average total network power with the data transfer time. 

\subsection{General SimGrid related setup}
For all of our experiments we used a simple client/server SimGrid application. As we have described in Chapter~\ref{chapter:environment}, a typical SimGrid script contains three sections. Accordingly in our script, the first section represents what the client function do to send data. It simply accepts the number of bytes to send from command line option or use a default 1000 bytes if no value is specified, or it will generate the bytes if random option is specified in the command line. Then pass the bytes to the SimGrid's send routine. By default there will be only one TCP flow for the specified bytes, but if we want to send multiple flows we can specify the number of flows on the command line. The server function simply issues the receive routine if there is data to receive. The third section, in addition to doing the tasks specified in Chapter~\ref{chapter:environment},  it is also a gateway to NS-3 or our implemented flow-level model depending on the option specified in the command line. 

For the Link tag in SimGrid's platform file we used 10 Mbps  as a bandwidth value. This value is chosen because it falls within the range where ECOFEN's and SimGrid's simulated time value match closely as we have described by the end of Chapter~\ref{chapter:implementation}. For latency we used 10 ms and as a watt-range power consumption value, we used 10.3581 watt as idle power consumption value and 10.7479 watt as a busy power consumption value. 

SimGrid's interface to NS-3 will set these bandwidth and latency values, together with other parameters, to NS-3's configuration automatically. 

\section{Accuracy Validation}

For accuracy validation we did two sets of experiments. In the first set, our purpose is two investigate the accuracy of energy consumption estimation difference between ECOFEN and the flow-level model when the size of platform changes. In SimGrid case which means when the number of Links change. In ECOFEN case, when the number of Nodes, NetDevices and the connection between them change. For this experiment we kept the number of flows and the bytes size fixed but increase the number of Links. In the second set, on the other hand, our purpose is to study the difference between the estimated energy consumption value between ECOFEN and SimGrid when the data size or flow change while keeping the number of links fixed.

\subsection{Effect of platform size on accuracy}

\subsection{Effect of data size on accuracy}
  
\section{Scalability Validation}

\subsection{Runtime}

\subsection{Peak Memory usage}

\section{Simulating large-scale network}
