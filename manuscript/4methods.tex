\chapter{Methods}
\label{chapter:methods}
One approach for estimating energy consumption of a given network is by employing actual power meter to measure the power drawn by involved network and computing components. A good example for such case is the measurement that Fan and his team conducted\cite{DBLP:conf/isca/FanWB07}. In this study the authors have managed to monitor power consumption of several thousand of servers over a period of six months on real live workload. Mahadevan et al.{\ } have also done a similar power measurement on a production environment for studying power consumption behavior of networking devices such as switches and routers. If the measurements are done correctly, this approach produces the most real picture of the network under investigation compared to the other approaches discussed in subsequent paragraphs. However, this approach has certain inherent drawbacks. First real production networks might not be available for experimentation. Even if they become available, the transient and varying nature of the production environment makes it hard to repeat the experiment. Second we have little or no control over factors affecting the measured power consumption. We do not have the privileged of injecting or modifying the traffic or the workload in order to test different experimental hypothesis. To have a full control we need another approach. That is what we discuss next.

Experimental testbed is another approach that researchers have used to study power consumption characteristics of different computing and networking equipments. In this approach first a separate network is setup and configured solely for the purpose of conducting experiments. Then researchers can make measurements by manipulating factors that affect power consumption according to the hypothesis that they want to test. Unlike the previous one, in this approach the researcher have greater flexibility over the experimental parameters. In the power measurement study scenario that we are discussing, the researcher can change parameters such as traffic rate, packet size, inter-packet time interval and transmission protocol used (TCP/UDP). 