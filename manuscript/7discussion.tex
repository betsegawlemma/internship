\chapter{Discussion}
\label{chapter:discussion}

In this research our objective was to investigate the level of accuracy and scalability obtained from flow-level models for estimating energy consumption of large-scale network. In order to achieve this goal we layout a research method that describe the steps that we followed from start to finish. 

Following our method we started by studying literature to explore the state-of-the-art in the area of energy consumption of large-scale networks and the simulations available for estimating the consumption. From our literature study we have learned that the energy consumption property of a network equipment is characterized by two properties: idle power consumption and dynamic power consumption as a result of data transfer task. We have also learned that there are very few packet-level simulators proposed for estimating energy consumption of large-scale networks.

It is already known that packet-level simulators are more accurate in modeling a given network phenomenon compared to flow-level simulators but less scalable in terms of runtime and memory usage performance metrics \cite{DBLP:conf/infocom/LiuFGKT01}. The accuracy of packet-level simulators comes from the detail information they capture about simulated network phenomenon, they strive to capture packet-level details of the simulated network phenomenon. Flow-level simulators, on the other hand abstract away low-level details and model a given network phenomenon using analytical equations. Loosing low-level details allows flow-level models to scale well when the size of the simulated network increases. 

Due to this trade-off between accuracy and scalability between the two simulation approaches, in our research, we stated the following hypothesis in order to investigate the level of accuracy and scalability we can get from flow-level models.

``\emph{flow-level energy consumption models can give reasonably accurate estimation and they can also be significantly more scalable than packet-level models}``

In order to test this hypothesis, we first implemented a flow-level energy consumption model that we got from literature for SimGrid and then conducted accuracy and scalability experiments as we have described in Chapter~\ref{chapter:evaluation}. 

The result of the accuracy validation experiments show that in fact very good energy consumption estimation accuracy can be obtained using flow-level models. For all the five scenarios we run, the relative error we registered lies approximately between  0.1\% and 0.3\% compared to the packet-level model used in ECOFEN. Our unequal variance t-test statistical test with p-value around 0.9 also tells us that statistically the estimation difference between the packet-level simulator and the flow-level simulator is not significant. This confirms the first half of our hypothesis that flow-level models can give reasonably accurate estimation. The condition that should be satisfied in our model to let this hypothesis hold true is, correct time prediction, since our analytical equation (flow-level model) uses simulated time as one of its parameter. 

During our experimentation, we have noticed that there is significant difference between ECOFEN (or NS-3) and SimGrid on predicting the simulated time required to transfer a given data. For a given mega bytes of data, at a given latency value, SimGrid's predicted time continues to decrease as bandwidth increases while ECOFEN stops to decrease approximately when the bandwidth goes above 128 Mbps as shown in Figure~\ref{fig:bandwidthvstime}. Since addressing this problem is beyond the scope of our work, the approach we followed to avoid estimation error due to this time discrepancy is that we restricted all our experiment to bandwidth values where the two simulators closely agree on predicted time value.

Another point we like to point out about our accuracy validation experiment is that, there are additional tests that should be conducted to validate the correctness of the implemented model in different data transfer scenarios. For example SimGrid support simulating simultaneous TCP flows in same or different directions. All of the flows that we have used in our experiments were in the same direction. 

Concerning the scalability, the runtime and peak memory usage experiment results confirmed the second half of our hypothesis which states that flow-level models are significantly scalable than their packet-level counter part. In the runtime performance metrics we have shown in Table~\ref{table:runtime} that SimGrid is 243 to 2723 times faster than ECOFEN. Similarly in Table~\ref{table:peakmemory} we have shown that SimGrid is 2 to 15 times more memory efficient than ECOFEN. Actually the scalability performance did not come directly from our implementation. It is due to the scalability of the underlying flow-level model of SimGrid. The scalability of SimGrid is already confirmed in other research works performed in the area of large-scale distributed systems~\cite{DBLP:conf/ccgrid/QuinsonRT12,DBLP:journals/jpdc/CasanovaGLQS14}.

Two of the limitations of our work is related to SimGrid's TCP model. The first one is that the TCP model in SimGrid can work in both half-duplex and full-duplex mode, however, our implementation works only in half-duplex mode. The second one  is that SimGrid uses different bandwidth sharing options that determine how the bandwidth is shared among the flows traversing a given link. In our implementation, we have only considered the option which share bandwidths fairly among the flows. 

The other limitations of our work come from our scope. We have first limited our study to communicating components of large-scale network. There are other energy consuming components such as data-center infrastructure facilities (which includes power provisioning, cooling and lighting components) that should be modeled in order to give full energy consumption estimation of a given large-scale network. Our work was also limited to the wired network components. As a result we have not considered any communicating components that are involve solely on the wireless network such as Wi-Fi access points. 

Using our implementation together with already existing power consumption model of SimGrid for CPUs, it is possible to simulate energy consumption of computing and communicating components of large-scale networks. However, using present implementation, it is only possible to experiment on one kind of energy saving techniques: switching links ON/OFF. One can experiment on the effect of switching network components on or off to study the resulting energy cost difference. 

Other experiments, such as, investigating the effect of activating adaptive link rate energy saving mode on a link, can not be conducting using the current implementation. However, the current implementation can be extended to allow this feature by following the P-state approach implemented in the existing CPU energy consumption model. Similar to CPU's P-state, a link can have multiple data transmission rate levels. 

We like to recommend three areas for future researchers who like to extend our work. First, to include in our model other features of SimGrid's TCP model such as full-duplex and different bandwidth sharing options.  Second, to add more energy saving feature such as adaptive link rate. Third, to propose and implement flow-level energy consumption model for wireless network components following the method that we outline. 
