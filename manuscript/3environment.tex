\chapter{Environment}
\label{chapter:environment}
In this study we employed SimGrid, ECOFEN, FlowMonitor modules of NS-3 simulator and other tools. This chapter explains the main features of these tools from the perspective of our simulation experiment needs.

\section{SimGrid}
\label{section:simgridenvironment}
In Section~\ref{section:simgrid} of Chapter~\ref{chapter:background} we discussed the software architecture of SimGrid at a higher level. We will give low-level details of the implemented flow-level model and related concepts in later chapter. In this section, our plan is to discuss features of SimGrid that are related to setting up and running energy consumption experiments.

In Figure~\ref{fig:SimGrid} of Chapter~\ref{chapter:background} we have presented three user APIs that SimGrid users can use to develop their simulation experiments. Currently there is another API named S4U that is under development. This API is similar in usage to MSG API. One main difference, from users perspective, is that MSG is in C while S4U is in C++ language. S4U is the API that we have used in this study.  
  
Designing and running simulation experiments in SimGrid using MSG or S4U APIs involve creating three files: a C/C++ simulation Script, an XML file for specifying the simulated platform topology and another XML file for specifying the deployment options, such as, identifying the host that send or receive data, the size of data, and the number of processes sending the data. 

In a typical simulation experiment of energy estimation as a function of data transfer, we can have three sections in the simulation script. In the first section we write a function which specify what the sender do to send the data, in the second section we write what the receiver do to receive the simulated data or what action to take when the simulated data arrives. In the third section we tell to SimGrid's simulation engine about the two functions and we also pass to the engine the platform and the deployment file names. 

In SimGrid, simulated network resources such as, NICs, switches and routers are represented with an abstraction called Link. In SimGrid platform file we represent a Link as follows:

\begin{lstlisting}
<link id="SWITCH1" bandwidth="100MBps" latency="10ms">
      <prop id="watt_range" value="305:550"/>
</link>
\end{lstlisting}

From the link XML tag we can see the basic characteristics of a simulated switch, such as, bandwidth and latency. We also see the idle and busy power consumption range of the switch. Other additional information can also be added such as how the link should be shared when multiple traffic cross the link and tracing information to control the bandwidth and latency property while the simulation is progressing. 

In a similar manner we can specify deployment information such as the role of hosts, the number of processes and the size of transfered data as follows. 
\begin{lstlisting}
<process host="H-1" function="sender">
<argument value="10" />
</process>
<process host="H-2" function="receiver"/>
\end{lstlisting}
This separation of concern among the simulation script, the platform and the deployment configuration files offers great flexibility for designing and running large scale experiments. The platform can be scaled up or down without changing the simulation script, for instance. 

Another feature of SimGrid that we would like to mention here is that SimGrid provides access to NS-3 simulator. This has at least two main advantages. The first one is that for SimGrid users who like to have low level packet information about the simulated network phenomenon, they can launch their experiment from SimGrid interface while using their platform and deployment files that they have created within SimGrid. The second one is that for studies similar to ours this feature helps a lot during the validation process of newly implemented model. This feature allowed us to run the validation comparisons with NS-3 using the same platform and deployment file that we have created in SimGrid. SimGrid automatically maps the topology and the corresponding parameters into NS-3's abstraction.
\section{NS-3}
NS-3 is a discrete-event packet-level simulator, events corresponding to, for instance, arrival and departure of packets. NS-3 is structured in a modular manner. The core and the network modules are two of the modules that serve as generic simulation core that can be used for Internet-based or different network type simulation. These two modules, being generic, are independent from any device models. The core module provides features such as tracing, callbacks, smart pointer, random variables, events and schedules. The network module consists components such as packets, node, addresses (e.g.,~IPv4 and MAC) and network devices. The components provided by the simulation core modules can be used to create other modules. This feature allows researchers to add their own models for the network phenomenon that they want to simulate. We will visit two of the modules that are constructed in this way in the next two subsections\cite{ns3}. 

The NS-3 core and other modules are built in C++ language as a set of libraries. The user can access these libraries in their main C++ program to configure the simulated topology and other simulator parameters. The libraries are also available as Python API for those researchers who prefer Python programming language. 

\subsection{ECOFEN Module}
ECOFEN is one of the two non-core NS-3 modules that we used in our experiments. We explained the power consumption simulation features provided by this module in the Related Simulators section of Chapter~\ref{chapter:background} and we will give detailed explanation about why and where we have used it in our Method chapter. In this section, we only give brief description about how it is related to NS-3 and how we have used it. 

NS-3 in its core provides an abstraction such as Node, Net Device, Channel and Application. A Node represents  network communication and computing devices (currently NS-3 do not have CPU abstraction) such as servers, switches and routers. To a Node a Net Device, which represent devices such as network interface card (NIC), can be attached. Two or more Nodes can be linked to each other through a Channel, which is a representation of Ethernet or Wi-Fi link. These three abstractions: Node, Net Device, and Channel, together they can be used to define the simulated network topology. Application, on the other hand, is an abstraction that represent user program that perform some simulated activity such as sending or receiving UDP packets~\cite{ns3}. 

Using the core abstractions provided by NS-3, such as  Node, Net Device, and Packet, the ECOFEN module implemented three power consumption models that enable users to simulate power consumption as a consequence of packets transmission at different levels of granularity as discussed in Chapter~\ref{chapter:background} and Chapter~\ref{chapter:methods}.

In a typical NS-3 simulator script, in its main function, we can recognize four common sections: (1) the section where we find statements that import the required core or other modules, (2) the section where the topology of the simulated network is defined, (3) the section where the simulated user application is defined, and (4) the section where statements related to running, starting, stopping and cleaning the simulation is specified. This is a rough approximation, certainly there are other statements such as those that are related to logging and tracing. 

The NS-3 scripts that we used in our power consumption simulation experiments imported the ECOFEN module in their first section, set up and configured the energy consumption models in the second section, configured an application that send and receive UDP packets in the third section, and finally, in the fourth section, we stated when the simulation should start and end. 

\subsection{FlowMonitor Module}
FlowMonitor is the other non-core NS-3 module that we have employed in our study. This module is designed with the aim of providing generic network traffic inspection facility. It provides researchers, who want to measure the simulated network efficiency, with standard performance metrics such as bit-rate, duration, delay, packet-size and packet loss ratio~\cite{DBLP:conf/valuetools/CarneiroFR09}.  

Among the performance metrics that are available in FlowMonitor module, the following are the ones that we have used in our simulation experiments.
\begin{itemize}
	\item \textbf{\textit{rxBytes}} to get the received bytes by a node,
	\item \textbf{\textit{txPackets}} to get the transmitted packets by a node,
	\item \textbf{\textit{timeFirstRxPacket}} and \textbf{\textit{timeLastRxPacket}} to get the absolute time when the first packet and the last packets in the flow was received,
	\item  \textbf{\textit{timeFirstTxPacket}} and \textbf{\textit{timeLastTxPacket}} to get the absolute time when the first and last packets in the flow was transferred and 
	\item \textbf{\textit{lostPackets}} to check if there are lost packets.
\end{itemize}
We used the above performance metrics to compute throughput (T) with the unit of Mega-bits per second (Mbps) and Packets per second(Pps) as shown in Equation~\ref{eq:3.1} and Equation~\ref{eq:3.2}. 
\begin{equation} \label{eq:3.1}
\begin{split}
T_{Mbps} &= rxBytes \times 8.0 \times 10^{-6} /\\
  & (timeLastRxPacket - timeFirstRxPacket)
\end{split}
\end{equation}
\begin{equation} \label{eq:3.2}
T_{Pps} = txPackets / (timeLastTxPacket - timeFirstTxPacket)
\end{equation}
\section{Other tools}
We followed, partly, the literate programming and reproducible research approach proposed in~\cite{DBLP:journals/sigops/StanisicLD15,schulte2012multi}. In this approach, the authors used two well-known tools: Git and Org-mode. A Git branching model is proposed in~\cite{DBLP:journals/sigops/StanisicLD15} that ease the synchronization of data and the code that generated the data. Org-mode, on the other hand, is employed as a literate programming tool for managing a laboratory notebook.  

Org-mode\footnote{http://orgmode.org/} is a plain text mark-up language which is available as an extension to Emacs text editor. An Org-mode document can have different sections: a plain text, an executable code block and/or a data block. The code and the data blocks are active, meaning they can be evaluated (or executed) and as a result they can output the code or the data block as passive (plain text) form and/or the computational result of the evaluated code or the data block. This feature allows Org-mode to be a powerful tool for literate programming.

Whenever we wanted to do some experiment, we used Org-mode in our laboratory notebook to capture the experimental environment, such as, the objective of the experiment, the assumptions we made, the parameters used, the links referenced, and any other information relevant to our experiment. Within the same document we also put chunks of codes wherever we want them using programming languages such as bash shell, python, and R. What is more amazing is that Org-mode allowed us to name and call the executable codes from anywhere within the document with different input parameters, hence we were able to reuse previously written code blocks. 

We used the shell script to run NS-3 and SimGrid simulation scripts and also to capture their outputs. We used Python to extract and format the data we want from the raw data produced by the simulators. Then we used R with its ggplot2 package to generate different plots and to do statistical analysis. All the experiments we conducted, the statistical analysis we made and the plots we generated can be reproduced by anyone. The lab notebook is available at our github account\footnote{https://github.com/betsegawlemma/internship/blob/master/intern{\_}report.org}.


